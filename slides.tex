\documentclass{beamer}
\usetheme{metropolis}
\usepackage{wasysym}
\usepackage{ebproof}

\newcommand{\pity}[3]{\prod_{(#1~:~#2)} #3}
\newcommand{\sigmaty}[3]{\sum_{(#1~:~#2)} #3}
\newcommand{\univ}{\mathcal{U}}
\newcommand{\rulename}[1]{\textsf{\color{blue} #1}}
\newcommand{\pow}[1]{\mathcal{P}\left( #1 \right)}

\title{Pointless Topology in Univalent Foundations}

\date{\today}
\author{Ayberk Tosun}
\institute{Chalmers University of Technology}

\begin{document}

\maketitle

%% Slide 1.
\begin{frame}{Motivation}
  \begin{align*}
    \text{{\huge Top}}&\text{{\huge ology}}\\
    \text{\alert{understood}} &~\text{\huge $\Downarrow$}~\text{\alert{constructively}} \\
    \text{{\huge Pointles}}&\text{{\huge s~topology}}\\
    \text{\alert{understood}} &~\text{\huge $\Downarrow$}~\text{\alert{predicatively}} \\
    \text{{\huge Forma}}&\text{{\huge l~topology}}\\
  \end{align*}
\end{frame}

%% Slide 2.
\begin{frame}{Frames}
  A poset $\mathcal{O}$ such that
  \begin{itemize}
    \item \alert{finite subsets} $\mathcal{O}$ have \alert{meets},
    \item \alert{all subsets} of $\mathcal{O}$ have \alert{joins}, and
    \item binary meets distribute over arbitrary joins:
      \begin{equation*}
        A \wedge \left( \bigvee_{i~:~I} \mathbf{B}_i \right) = \bigvee_{i~:~I} \left( A \wedge \mathbf{B}_i \right),
      \end{equation*}
      for any $A \in \mathcal{O}$ and $I$-indexed family $\mathbf{B}$ over $\mathcal{O}$.
  \end{itemize}
\end{frame}

%% Slide 3.
\begin{frame}{Presenting Frames: Interaction Structure}
  Take a poset equipped with an \alert{interaction structure}:
  \begin{align*}
    A ~&:~ \univ{}_m \\
    \sqsubseteq ~&:~ A \rightarrow A \rightarrow \mathsf{hProp}_n\\
    B ~&:~ A \rightarrow \univ{}_m\\
    C ~&:~ \pity{x}{A}{B(x) \rightarrow \univ{}_m}\\
    d ~&:~ \pity{x}{A}{\pity{y}{B(x)}{C(x, y) \rightarrow \univ{}_m}}
  \end{align*}
  that satisfies certain axioms.

  This is our notion of a \alert{formal topology}.
\end{frame}

\begin{frame}{Presenting Frames: The Coverage Relation}
  We define the following \alert{coverage} relation on it $\LHD$
  \[
    \begin{prooftree}
      \hypo{ a~\epsilon~U }
      \infer1[\rulename{dir}]{a \LHD U}
    \end{prooftree}
    \qquad
    \begin{prooftree}
      \hypo{d(a, b, c) \LHD U \rightarrow a \LHD U}
      \infer1[\rulename{branch}]{a \LHD U}
    \end{prooftree}
  \]
  which we cannot show to be propositional.
\end{frame}

%% Slide 5.
\begin{frame}{Presenting Frames: $\LHD$ \emph{must} be propositional}
  To show that our formal topologies present frames, we want to treat the coverage itself
  as forming a predicate.

  \vspace{2em}

  \begin{center}
    \begin{tabular}{l l}
    We have                          & We want                                          \\
    $\LHD : A \rightarrow \pow{A} \rightarrow \univ{}_m$ & $\LHD : A \rightarrow \pow{A} \rightarrow \mathsf{hProp}_m$          \\
    $\RHD : \pow{A} \rightarrow A \rightarrow \univ{}_m$ & $\RHD : \pow{A} \rightarrow \pow{A}$
    \end{tabular}
  \end{center}
\end{frame}

%% Slides 5'.
\begin{frame}{Presenting Frames: $\LHD$ \emph{must} be propositional}
  To show that our formal topologies present frames, we want to treat the coverage itself
  as forming a predicate.

  \vspace{2em}

  \begin{center}
    \begin{tabular}{l l}
    We have                          & We want                                          \\
    $\LHD : A \rightarrow \pow{A} \rightarrow \univ{}_m$ & $\LHD : A \rightarrow \pow{A} \rightarrow \mathsf{hProp}_m$          \\
    $\RHD : \pow{A} \rightarrow A \rightarrow \univ{}_m$ & \fbox{$\RHD : \pow{A} \rightarrow \pow{A}$}
    \end{tabular}
  \end{center}
\end{frame}

%% Slide 6.
\begin{frame}{First Frame}
  Hello, world!
\end{frame}

\end{document}
